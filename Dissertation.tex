\documentclass[12pt,a4paper]{article}
\usepackage{titlesec} %these are how we import packages, one helps set up footers and title layout
\usepackage{fancyhdr}
\usepackage{lipsum}
% !TEX TS-program = pdflatex
% !TEX encoding = UTF-8 Unicode
\usepackage[utf8]{inputenc} % set input encoding (not needed with XeLaTeX)
\usepackage[left=1.18in,top=0.95in,right=0.79in,bottom=0.79in]{geometry}
\usepackage{graphicx} % support the \includegraphics command and options

% \usepackage[parfill]{parskip} % Activate to begin paragraphs with an empty line rather than an indent

%%% PACKAGES
\usepackage{booktabs} % for much better looking tables
\usepackage{array} % for better arrays (eg matrices) in maths
\usepackage{paralist} % very flexible & customisable lists (eg. enumerate/itemize, etc.)
\usepackage{verbatim} % adds environment for commenting out blocks of text & for better verbatim
\usepackage{subfig} % make it possible to include more than one captioned figure/table in a single float
\usepackage[toc,page]{appendix}
% These packages are all incorporated in the memoir class to one degree or another...

%header and footer settings
\pagestyle{fancyplain}
\fancyhf{}
\renewcommand{\sectionmark}[1]{\markboth{#1}{}} % set the \leftmark
\renewcommand{\headrulewidth}{0.5pt}
\renewcommand{\footrulewidth}{0.5pt}
\setlength{\headheight}{15pt}
\fancyhead[L]{Sam Serrels - 40082367}
\fancyhead[R]{ SOC10101 Honours Project}
\fancyfoot[L]{\leftmark}
\fancyfoot[C]{\thepage}

%set better section layout
\makeatletter
\renewcommand\subsection{\@startsection {subsection}{1}{2mm} % name, level, indent
                               {15pt} % before skip
                               {8pt} % after skip
                               {\fontsize{13pt}{1em}\bfseries}}
\makeatother
\makeatletter
\renewcommand\section{\@startsection {section}{1}{0mm} % name, level, indent
                               {0pt} % before skip
                               {20pt} % after skip
                               {\fontsize{14pt}{1em}\bfseries\newpage}}
\makeatother


%this starts the document
\begin{document}

%you can import other documents into your main one, these layout the Title and Declarations on its own page.
%you might need to change these to \ if your on Microsoft Windows.
\input{./Dissertation-Title.tex}
\input{./Dissertation-Dec.tex}
\pagebreak
\input{./Dissertation-DP.tex}
\pagebreak

%LaTeX let you define the abstract separately so it wont get sucked into the main document.
\begin{abstract}
This project aims to research the viability of compressing data on a Graphics Processing Unit(GPU) before sending it to another GPU to reduce the transfer time and bandwidth utilisation.
The resulting data will be used to analyse the suitability of implementing compression methods into existing GPU workloads, such as real-time rendering, or distributed general purpose computation.
\end{abstract}
\pagebreak

\tableofcontents % is generated for you
\newpage

\listoftables
%generated in same way as figures
\newpage

\listoffigures
%you may have captions such as equations, listings etc they should all appear as required
%these are done for you as long as you use \begin{figure}[placement settings] .. bla bla ... \end{figure}
\newpage

%\section*{Acknowledgements}
%Insert acknowledgements here
%\subsection*{}
%	I would like to thank my cat, dog and family.
%\newpage

\section{Introduction}
	\subsection{Aims and Objectives}
		\lipsum[5]
	\subsection{Research Questions}
		\lipsum[8]
	\subsection{Scope}
		\lipsum[7]
	\subsection{Report Structure}
		\lipsum[6]
		
\section{Background}
	\lipsum[2]
	\subsection{Parralel alogrithms}
		\lipsum[3]
	\subsection{Graphics Processing Units}
		\lipsum[4]
	\subsection{GPGPU APIs}
		\lipsum[2]
	\subsubsection{OpenCl}
		\lipsum[2]
	\subsubsection{Cuda}
		\lipsum[2]
	\subsection{Data Transfer bottlenecks}
		\lipsum[3]
	\subsection{Data Compression}
		This is a sub sub section with a list of bullet points.
		\begin{itemize}\itemsep0pt
			\item A working X, that will be used for this investigation.
			\item Investigation of current tools and their potential use during an investigation of X .
			\item Programming of X with related frameworks Y and Z.
			\item That is all.
		\end{itemize}

\section{Literature Review}
	\subsubsection{Data tWrasfer between nodes in a muti-node system}
		Distributed systems are commonplace in solving large computation tasks.\\
		Not just HPC - general networking, and single pc system\\
		Time is money\\
		Waiting on transfers is a waste of money and energy\\
		Trade-off between adding more nodes verses fewer faster nodes, often super fast anyway\\
	\subsubsection{Data Compression}
		Basic method of not sending data that doesn't need sending \\
		Lossless is always needed expect in some circumstances\\
		Hardware level compression is a thing\\
		Sequential vs parallel\\
	\subsubsection{Gpu Compression}
		Double vs Single precision \\
		No Branching \\
		Gpus can already compress iamges quickly\\
	
	\subsubsection{Gpu Data Transfer}
		busy Cpu can cause slowdowns\\
		Busy PCI device can cause slowdowns\\
		Constant bandwidth, even with more devices\\
		Hardware DMa\\
		custom firmaware\\
		Api abstraction\\

\section{Test Framework}

\newpage

\bibliographystyle{plain}
\bibliography{Dissertation}

\newpage
\begin{appendices}
\section{Project Overview}
%insert IPO

\begin{subappendices}
\subsection{Example sub appendices}
...
\end{subappendices}

\section{Second Formal Review Output}
Insert a copy of the project review form you were given at the end of the review by the second marker

\section{Diary Sheets (or other project management evidence)}
Insert diary sheets here together with any project management plan you have

\section{Appendix 4 and following}
insert content here and for each of the other appendices, the title may be just on a page by itself, the pages of the appendices are not numbered, unless an included document such as a user manual or design document is itself pager numbered.
\end{appendices}

\end{document}
